\documentclass[a4paper,12pt]{article}
\title{LVM - Logic Volume Management}
\author{Fabio Rampoldi}
\usepackage[italian]{babel} 
\usepackage[utf8x]{inputenc}
\begin{document}
\maketitle
\pagebreak 
\tableofcontents
\pagebreak 

\section{Presentazione del LVM}
\subsection{Principi}
Il LVM, \textit{Logical Volume Management} o gestione di volumi logici, è un metodo che descrive una gestione logica dello spazio della memoria di massa di un computer. All'inizio creato per i fabbisogni dei servers, è attualmente utilizzato su Linux come metodo di gestione dello spazio degli harddisk di default nelle principali distribuzioni.

Il principio di base consiste nella concatenazione di più spazi 
\subsection{Composizione}
\subsection{Estensioni}
\subsection{File di configurazione}


\end{document}